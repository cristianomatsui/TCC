\begin{quadro}[!htb]
    %\centering
    \caption{Cronograma de Atividades.\label{qua:cronograma}}
    \begin{tabular}{|p{5.0cm}|p{0.6cm}|p{0.6cm}|p{0.6cm}|p{0.6cm}|p{0.6cm}|p{0.6cm}|p{0.6cm}|p{0.6cm}|p{0.6cm}|p{0.6cm}|}
        \hline
        \textbf{Atividades} 						     & 	   \textbf{Set}      & \textbf{Out} & \textbf{Nov} & \textbf{Dez} & \textbf{Jan} & \textbf{Fev} & \textbf{Mar} & \textbf{Abr} & \textbf{Mai} & \textbf{Jun} \\
        \hline
        \footnotesize{1. Estudo e acompanhamento da literatura}		     & X	             & X            &  		   & 		  &  		 &   	        &   	       &   	      & 	     & 		     \\
        \hline
        \footnotesize{2. Modelagem do problema} 				     & X	             & X            &  		   & 		  &  		 &   	        &   	       &   	      & 	     &                \\
        \hline
	\footnotesize{3. Verificação das estruturas Omni} 			     &  	             & X            & X		   & 		  &  		 &   	        &   	       &   	      & 	     &                \\
        \hline
	\footnotesize{4. Estudo dos extratores de características} 		     &  	             &              & X		   & 		  &  		 &   	        &   	       &   	      & 	     & 		     \\
        \hline
	\footnotesize{5. Estudo e inserção das métricas no banco} 		     &  	             & X            & X		   &  		  &  		 &   	        &   	       &   	      & 	     & 		     \\
        \hline
	\footnotesize{6. Criação e povoamento da base de dados} 		     &  	             &              & X		   & X		  &  		 &   	        &   	       &   	      & 	     & 		     \\
        \hline
        \footnotesize{7. Elaboração da apresentação do TCC1}			     &  	             &              & X		   & X		  &  		 &   	        &   	       &   	      & 	     & 		     \\
        \hline        
        \footnotesize{8. Defesa do TCC1}                                             &  	             &              &  		   & X		  &  		 &   	        &   	       &   	      & 	     & 		     \\
        \hline
	\footnotesize{9. Implementação da busca sequencial} 			     &  	             &              &  		   & 		  & X		 & X	        & X 	       &   	      & 	     & 		     \\
        \hline
        \footnotesize{10. Implementação da técnica Omni} 			     &  	             &              &  		   & 		  & X		 & X 	        & X 	       &   	      & 	     & 		     \\
        \hline
        \footnotesize{11. Comparação e análise de resultados} 			     &  	             &              &  		   & 		  &  		 &   	        &  X	       & X 	      & 	     & 		     \\
        \hline
        \footnotesize{12. Implementação da interface} 				     &  	             &              &  		   & 		  &  		 &   	        &  X	       &  X	      & 	     & 		     \\
        \hline
        \footnotesize{13. Elaboração da apresentação final} 			     &  	             &              &  		   & 		  &  		 &   	        &   	       &  X	      &  X	     & 		     \\
        \hline        
        \footnotesize{14. Defesa do TCC2}					     &  	             &              &  		   & 		  &  		 &   	        &   	       &   	      & 	     & X	     \\
        \hline
    \end{tabular}
\end{quadro}

%Estudo, modelagem, verificação das estruturas Omni, bases, extratores, métricas, criar o banco e inserir as métricas, implementação da Omni, implementação seq sem Omni, comparação