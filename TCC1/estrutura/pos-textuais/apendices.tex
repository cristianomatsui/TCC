% APÊNDICES--------------------------------------------------------------------

\begin{apendicesenv}
\partapendices

% Primeiro apêndice------------------------------------------------------------
\chapter{SCRIPTS OMNI} % Edite para alterar o título deste apêndice
\label{chap:apendice}

Neste apêndice, estão alguns dos demais scripts e códigos utilizados para a criação das estruturas OMNI, assim como 
os scripts SQL utilizados para a criação das tabelas utilizadas pelo SGBD. Foram descritos aqui apenas scripts
referentes à base CAT\_DOG. Os scripts OMNI para a base HC são virtualmente os mesmos, alterando apenas o nome
de algumas tabelas para as tabelas análogas da outra base.

\begin{lstlisting}[caption={Criação das tabelas primárias do banco}, captionpos=t,basicstyle=\tiny]
CREATE TABLE COMPLEX_DATA (
ID_CD VARCHAR(15) PRIMARY KEY);

CREATE TABLE SHAPE (
ID_FT INTEGER PRIMARY KEY,
CD_REF VARCHAR(15),
IS_FOCUS BOOLEAN DEFAULT 'FALSE',
FEATURE FLOAT8[],
CONSTRAINT CD_FK FOREIGN KEY (CD_REF) REFERENCES COMPLEX_DATA (ID_CD)
ON DELETE CASCADE ON UPDATE CASCADE); 

CREATE TABLE COLOR (
ID_FT INTEGER PRIMARY KEY,
CD_REF VARCHAR(15),
IS_FOCUS BOOLEAN DEFAULT 'FALSE',
FEATURE FLOAT8[],
CONSTRAINT CD_FK FOREIGN KEY (CD_REF) REFERENCES COMPLEX_DATA (ID_CD)
ON DELETE CASCADE ON UPDATE CASCADE); 

CREATE TABLE TEXTURE (
ID_FT INTEGER PRIMARY KEY,
CD_REF VARCHAR(15),
IS_FOCUS BOOLEAN DEFAULT 'FALSE',
FEATURE FLOAT8[],
CONSTRAINT CD_FK FOREIGN KEY (CD_REF) REFERENCES COMPLEX_DATA (ID_CD)
ON DELETE CASCADE ON UPDATE CASCADE); 
\end{lstlisting}

\begin{lstlisting}[caption={Criação das tabelas OMNI}, captionpos=t,basicstyle=\tiny]
CREATE TABLE SHAPE_F_BASE(
ID_FOCUS INTEGER,
ID_FEATURE INTEGER,
DIST_L1 FLOAT8,
DIST_L2 FLOAT8,
DIST_LINF FLOAT8,
PRIMARY KEY (ID_FOCUS, ID_FEATURE),
CONSTRAINT SHAPEFOCUS_FK FOREIGN KEY (ID_FOCUS) REFERENCES SHAPE (ID_FT)
ON DELETE CASCADE ON UPDATE CASCADE,
CONSTRAINT SHAPEFEATURE_FK FOREIGN KEY (ID_FEATURE) REFERENCES SHAPE(ID_FT)
ON DELETE CASCADE ON UPDATE CASCADE);

CREATE TABLE COLOR_F_BASE(
ID_FOCUS INTEGER,
ID_FEATURE INTEGER,
DIST_L1 FLOAT8,
DIST_L2 FLOAT8,
DIST_LINF FLOAT8,
PRIMARY KEY (ID_FOCUS, ID_FEATURE),
CONSTRAINT COLORFOCUS_FK FOREIGN KEY (ID_FOCUS) REFERENCES COLOR (ID_FT)
ON DELETE CASCADE ON UPDATE CASCADE,
CONSTRAINT COLORFEATURE_FK FOREIGN KEY (ID_FEATURE) REFERENCES COLOR(ID_FT)
ON DELETE CASCADE ON UPDATE CASCADE);

CREATE TABLE TEXTURE_F_BASE(
ID_FEATURE INTEGER,
ID_FOCUS INTEGER,
DIST_L1 FLOAT8,
DIST_L2 FLOAT8,
DIST_LINF FLOAT8,
PRIMARY KEY (ID_FOCUS, ID_FEATURE),
CONSTRAINT TEXTUREFOCUS_FK FOREIGN KEY (ID_FOCUS) REFERENCES TEXTURE (ID_FT)
ON DELETE CASCADE ON UPDATE CASCADE,
CONSTRAINT TEXTUREFEATURE_FK FOREIGN KEY (ID_FEATURE) REFERENCES TEXTURE(ID_FT)
ON DELETE CASCADE ON UPDATE CASCADE);
\end{lstlisting}

\begin{lstlisting}[caption={Inserção das distâncias na tabela de focos}, captionpos=t,basicstyle=\tiny]
CREATE OR REPLACE FUNCTION insertShapeDistance() 
RETURNS VOID AS $$
BEGIN
  INSERT INTO shape_f_base (id_focus, id_feature, dist_l1, dist_l2, dist_linf)(
  select S1.id_ft, S2.id_ft, cube(S1.feature) <#> cube(S2.feature),
  cube(S1.feature) <->  cube(S2.feature), cube(S1.feature) <=>  cube(S2.feature)
  from shape S1, shape S2 
  where S1.is_focus = TRUE AND S1 <> S2) ;
END;
$$ LANGUAGE plpgsql;  
\end{lstlisting}

\begin{lstlisting}[caption={Exclusão da base focal}, captionpos=t,basicstyle=\tiny]
CREATE OR REPLACE FUNCTION wipe_shape_focus_base () 
RETURNS VOID AS $$
BEGIN
  update shape set is_focus = false where is_focus = true;
  delete from shape_f_base;
  RAISE NOTICE 'Shape Focus Base wiped';
END;$$
LANGUAGE PLPGSQL;
\end{lstlisting}

\begin{lstlisting}[caption={Criação da base focal OMNI}, captionpos=t,basicstyle=\tiny]
CREATE OR REPLACE FUNCTION createShapeFocusBase(num integer)
RETURNS VOID AS $$
DECLARE fprev_id integer; fnext_id integer; distance float8; 
n_inserted integer := 0; border float8;
BEGIN
  PERFORM wipe_shape_focus_base();
  select id_ft into fprev_id from shape  -- SELECIONA UM DADO RANDOM
  order by random()
  LIMIT 1;
  LOOP
    select S2.id_ft, cube(S1.feature) <-> cube(S2.feature) as dist 
    into fnext_id, distance from shape S1, shape S2
    where S1.id_ft = fprev_id AND S2.is_focus = 'False'
    order by dist DESC
    LIMIT 1;

    update shape
    set is_focus = 'True'
    where id_ft = fnext_id;

    n_inserted = n_inserted + 1;
    num = num - 1;
    RAISE NOTICE 'Num = %, n_inserted = %', num, n_inserted;
    EXIT WHEN num = 0;
    IF n_inserted = 2 THEN
      select cube(S1.feature) <-> cube(S2.feature) into border 
      from shape S1, shape S2  --CALCULA O VALOR DA BORDA
      where S1.id_ft = fprev_id and S2.id_ft = fnext_id;
      fprev_id = fnext_id;

      LOOP					
	select S2.id_ft, abs(border - (cube(S1.feature) <-> cube(S2.feature))) 
	as err into fnext_id, distance from shape S1, shape S2
	where S1.id_ft = fprev_id AND S2.is_focus = 'False'
	order by err ASC	-- minimiza o erro
	LIMIT 1;

	update shape
	set is_focus = 'True'
	where id_ft = fnext_id;

	fprev_id = fnext_id;					
	num = num - 1;
	EXIT WHEN num <= 0;
      END LOOP;
      EXIT WHEN num = 0;
    END IF;

    fprev_id = fnext_id;
    END LOOP;
  PERFORM insertShapeDistance();
END;$$
LANGUAGE PLPGSQL;
\end{lstlisting}

\begin{lstlisting}[caption={Range Query sequencial - L2}, captionpos=t,basicstyle=\tiny]
CREATE OR REPLACE FUNCTION rangeQueryShapeL2 (center_id integer, radius FLOAT8) RETURNS SETOF genericQuery AS $$
BEGIN
  RETURN QUERY SELECT DISTINCT * FROM 
  (SELECT T2.id_ft, (cube(T1.feature) <-> cube(T2.feature)) 
  AS dist FROM SHAPE T1, SHAPE T2 
  WHERE T1.id_ft = center_id) 
  AS quer WHERE dist <= radius ORDER BY dist;
END;$$
LANGUAGE PLPGSQL;
\end{lstlisting}
\newpage
\begin{lstlisting}[caption={Range OMNI com 5 focos - L2}, captionpos=t,basicstyle=\tiny]
CREATE OR REPLACE FUNCTION rangeOmniShapeL2F5 (center_id integer, radius FLOAT8) 
RETURNS SETOF genericQuery AS $$
DECLARE rec_id RECORD;feature_aux FLOAT8[]; 
distance FLOAT8; dist_fc FLOAT8[];
BEGIN
  select T1.feature into feature_aux FROM Shape T1 where T1.id_ft = center_id;
  dist_fc = array(select distinct dist_l2 from SHAPE_F_BASE where id_feature = center_id);
  FOR rec_id IN SELECT id_feature from SHAPE_F_BASE WHERE ((dist_l2 < (radius + dist_fc[1])) 
  AND (dist_l2 > dist_fc[1] - radius)) INTERSECT
  (SELECT id_feature from SHAPE_F_BASE WHERE ((dist_l2 < (radius + dist_fc[2])) 
  AND (dist_l2 > dist_fc[2] - radius))) INTERSECT
  (SELECT id_feature from SHAPE_F_BASE WHERE ((dist_l2 < (radius + dist_fc[3])) 
  AND (dist_l2 > dist_fc[3] - radius))) INTERSECT
  (SELECT id_feature from SHAPE_F_BASE WHERE ((dist_l2 < (radius + dist_fc[4])) 
  AND (dist_l2 > dist_fc[4] - radius))) INTERSECT
  (SELECT id_feature from SHAPE_F_BASE WHERE ((dist_l2 < (radius + dist_fc[5])) 
  AND (dist_l2 > dist_fc[5] - radius))) LOOP
    select (cube(feature_aux) <-> cube(feature)) INTO distance 
    FROM SHAPE T1 WHERE rec_id.id_feature = T1.id_ft ;
    IF (distance < radius) THEN
      RETURN NEXT (rec_id.id_feature, distance);			       
    end if;				
  END LOOP;
END;$$
LANGUAGE PLPGSQL;
\end{lstlisting}

\end{apendicesenv}