% CONCLUSÃO--------------------------------------------------------------------

\chapter{CONCLUSÃO}
\label{chap:conclusao}
No presente capítulo será abordada a conclusão do trabalho aqui escrito. Pontos fracos e pontos fortes da técnica utilizada, assim como cenários de melhor empregabilidade da técnica. Também serão discutidos possíveis aprimoramentos deste trabalho, na forma de
trabalhos ou pesquisas futuras.

\section{PERFORMANCE DA TÉCNICA}

Para as consultas por abrangência, a técnica OMNI conseguiu um ganho de performance significativo quando comparada com o método sequencial. A Tabela \ref{tab:performance1} mostra os ganhos de performance da consulta mais rápida em relação ao método
sequencial. O cálculo de ganho de performance (GP) percentual se dá por: $GP_\% = (T_o/T_s - 1)*100$, sendo $T_o$ o tempo da consulta OMNI mais rápida (quantidade de focos ótima) e $T_s$ o tempo da consulta sequencial.

\begin{table}[H]
    \centering
    \caption[Ganho de Performance - CAT\_DOG]{Ganho de Performance - CAT\_DOG.
    \label{tab:performance1}}
   % \begin{tabular}{l l l l}
   \begin{tabular}{l c}
        \toprule
            &GP (\%)\\
        \midrule
            L$_1$ - FORMA & 399,20 \\
            L$_2$ - FORMA & 378,24 \\
            L$_\infty$ - FORMA & 329,30 \\
            L$_1$ - TEXTURA & 468,91 \\
            L$_1$ - COR & 397,55 \\
        \bottomrule
    \end{tabular}
\end{table}

O ganho de performance para a base CAT\_DOG foi extremamente satisfatório, dado ao tamanho reduzido e baixa complexidade dos dados. Inicialmente estimou-se que a técnica só teria resultados significativos com bases massivas e com complexidade elevada, cenários
aonde a poda de cálculos desnecessários traria um maior benefício em questões de tempo de execução.

A base HC apresentou um ganho de performance ainda maior com a técnica OMNI, devido a sua elevada complexidade de dados e quantidade de tuplas a serem analisadas em cada consulta. Isto ilustra o potencial da técnica para bases ainda maiores, ou até mesmo para a manipulação
de \textit{Big Data}. A Tabela \ref{tab:performance2} contém os ganhos de performance da técnica na base HC.

\begin{table}[H]
    \centering
    \caption[Ganho de Performance - HC]{Ganho de Performance - HC.
    \label{tab:performance2}}
   % \begin{tabular}{l l l l}
   \begin{tabular}{l c}
        \toprule
            &GP (\%)\\
        \midrule
            L$_1$ & 1460,20 \\
            L$_2$ & 1304,04 \\
            L$_\infty$ & - \\
        \bottomrule
    \end{tabular}
\end{table}

Porém, a técnica tem as suas limitações, como visto na Seção \ref{sec:limit}. Enquanto consultas sequenciais apresentam pouca perda de desempenho com o aumento do raio da consulta por abrangência, as consultas OMNI são muito mais sensíveis às variações de raio.
Isso se deve ao fato de raios muito elevados causam um crescimento na \textit{mbOr}, o que acarreta em um número maior de alarmes falsos, aumentando a quantidade de cálculos de distância necessários na etapa de refinamento. Sob a luz dessas informações, é seguro dizer que 
a técnica OMNI só pode ser empregada viavelmente para consultas que retornem até no máximo cerca de 2\% do número de tuplas presentes na base, como ilustrado pelas Tabelas \ref{tab:limit1} e \ref{tab:limit2}.

Outro ponto interessante a ser notado é que a técnica apresenta uma certa tolerância em relação ao número de focos. Como visto na Seção \ref{sec:temp_cons}, técnica OMNI ainda consegue um desempenho melhor que o do método sequencial mesmo com um número não ótimo de focos.

\section{CUSTOS DA TÉCNICA}

A técnica OMNI prevê que as melhorias de performance em tempo de execução são possíveis através de um custo adicional de espaço em disco, e o tempo necessários
para a criação das estruturas OMNI utilizadas, como mencionado previamente na Seção \ref{sec:defomni}. A Tabela \ref{tab:custos} contém o espaço em disco
necessário para armazenar as tabelas utilizadas pelo banco, assim como as estruturas OMNI criadas. O espaço necessário está dividido
entre espaço em disco para armazenar os dados, e o espaço utilizado pelas estruturas de indexação.

\begin{table}[H]
    \centering
    \caption[Custos de armazenamento]{Custos de armazenamento.
    \label{tab:custos}}
   % \begin{tabular}{l l l l}
   \begin{tabular}{l c c}
        \toprule
             &Índices (MB)& Dados (MB)\\
        \midrule
            SHAPE & 1,62& 4,86 \\
            COLOR & 0,55& 2,83 \\
            TEXTURE & 0,55& 2,83 \\
            SHAPE\_F\_BASE & 36& 1,47 \\
            HC\_TABLE & 53& 1058 \\
            COLOR\_F\_BASE & 12& 13 \\
            TEXTURE\_F\_BASE & 24& 1,47 \\
            HC\_F\_BASE & 263& 1229 \\
        \bottomrule
    \end{tabular}
\end{table}

O custo de tempo de criação das estruturas OMNI variam entre 30 segundos até 36 minutos, dependendo do tamanho da base e número de focos
utilizados. Mas este custo pode ser ignorado devido ao fato de que o processo de criação destas estruturas é realizado apenas uma vez,
e que a base focal é invariante a operações de inserções.

\section{TRABALHOS FUTUROS}
\label{sec:trabalhosFuturos}

Como visto neste trabalho, a técnica OMNI possui um grande potencial em acelerar consultas por similaridade, mas as suas limitações restringem um pouco a sua usuabilidade. Dentre os aprimoramentos futuros para este trabalho, destacam-se estudos mais específicos
sobre os limites da técnica e como expandí-los. Outra melhoria importante seria um algoritmo mais eficiente de consultas por $k$-vizinhos mais próximos que possa ser utilizado com a técnica OMNI.

Outro trabalho interessante a nível de um comparativo qualitativo seria o de aprimorar a segmentação e extração de características de imagens, de modo a produzir consultas com um maior índice de acerto, assim
como consultas que consigam operar com mais de uma característica ao mesmo tempo, utilizando métodos de ponderação para cada característica, buscando mantendo o ganho de performance proposto pela OMNI.

