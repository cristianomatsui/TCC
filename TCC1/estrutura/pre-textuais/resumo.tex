% RESUMO--------------------------------------------------------------------------------

\begin{resumo}[RESUMO]
\begin{SingleSpacing}
    
% Não altere esta seção do texto--------------------%TODO ISSO AQUI É HARD CODED--------------------------
\imprimirautorcitacao. \imprimirtitulo. \imprimirdata. 26 f. \imprimirprojeto\ – \imprimirprograma, \imprimirinstituicao. \imprimirlocal, \imprimirdata.\\ 
%---------------------------------------------------------------------------------------
					
A necessidade de armazenamento de mídias cada vez maiores em termos de tamanho de armazenamento e complexas é uma tendência que aumentou consideravelmente
com os avanços da tecnologia e comunicação. Estes dados conhecidos como dados complexos exigem uma complexidade estrutural
de armazenamento e análise maior do que dados simples como palavras ou números, além de requererem operadores especiais
de consultas, como a consulta por abrangência (Rq) e a consulta aos k-vizinhos mais próximos (kNNq). Dentre o conjunto de dados complexos
destacam-se as imagens, que precisam ser comparadas de acordo com características extraídas como cor, forma ou textura. Esta
comparação é realizada na forma de um cálculo de distância entre o valor da característica da imagem central da consulta em relação
a todas as outras imagens da base de dados. O tempo de consulta pode aumentar significativamente com o aumento da base de dados. 
Para contornar o problema da maldição da cardinalidade, este trabalho tem como proposta aplicar uma técnica (Omni) utilizada para promover 
uma etapa de filtragem do número de imagens a terem as suas distâncias calculadas, evitando a comparação com toda a base de dados.
Após a modelagem e construção de um banco de dados que suporte esta técnica, será implementado um sistema de recuperação de imagens
(CBIR).\\

\textbf{Palavras-chave}: Dados Complexos. OMNI. CBIR. Rq. kNNq.

\end{SingleSpacing}
\end{resumo}

% OBSERVAÇÕES---------------------------------------------------------------------------
% Altere o texto inserindo o Resumo do seu trabalho.
% Escolha de 3 a 5 palavras ou termos que descrevam bem o seu trabalho 

%TODO PADRONIZAR TÉCNICAS OMNI/TÉCNICA OMNI